\documentclass[12pt]{article}
\usepackage[hmargin=2.0cm,vmargin=1cm]{geometry}
\usepackage[utf8]{inputenc}
\usepackage{graphicx}
\usepackage{float}
\usepackage{natbib}
\usepackage{amsmath}

\title{\begin{LARGE}
{Statistics class 1}
\end{LARGE}}

\author{Juan Nicol\'as Garavito Camargo}

\begin{document}
\maketitle

\textbf{HW1: Find a pick the random number generator that you like, and understand
how that random generation works., find a library that generate random
numer in your fav language and figure out what is the period,
clustering. Find the one you would like, same some plots! }

How to know a random generator makes random number:

1. Look at the distribution, it should be flat.
2. You dont want a correlation between any sequence of numbers. I dont
want clusters in a N-dimesional plot. 
3. Every time you take random steps N of numbers and you average thos
numbers, you will get a gaussian ditribution.
4. See D-Jnuth definition of random numbers, is in the slides.

\section{Berfords law:}

P(x) = posterior in any units, k=factor to change to other units.

\begin{equation}
P(kx) = f(k)P(x)
\end{equation}

\begin{equation}
\int P(x)dx = 1 -> \int P(kx') dx' = 1 = k \int P(kx')dx' = 1 -> \int
P(kx')dx' = 1/k
\end{equation}

\begin{equation}
\int f(k)P(x)dx = 1\k -> f(k) \int P(x)dx = 1/k -> f(k) = 1/k
\end{equation}

Difirencitating the above expressions:

\begin{equation}
P(kx) = P(x)/k -> d/dK P(kx) = - P(x)/k^2 
\end{equation}

\begin{equation}
x P'(kx) = - P(x)/k^2 = (k=1) xP'(x) = -P(x)
\end{equation}

\begin{equation}
x dP/dx = -P -> dP/P = - dx/x -> dlnP = -dLnX
\end{equation}

\begin{equation}
P(x)  C/x
\end{equation}

The likelihood that ...

Simon Newcomb notest this first

Benfords law in elections, Nigeria!

\textbf{In web page there is a paper proving Benfords law}

\section{How to make a random generator, Linear Congruential Methiods}

\begin{equation}
X_{n+1} = (a X_n + c)mod   m=modulus
\end{equation}

$0<a<m$ is the multiplier, $0<c<m$ is the increment.

This not works always, try $m=10$m $a=c=X_0=7$

1. The long sequence that you can have before it repreats is $<m$.

Then you want to pick a very long value of $m$. What is this largest
number, in our computers if it is double precision 4bits, 8bytes, 2^32, this is a small
number. (why?)

$m=2^{32}$ if you want your random generator numbers you would like
$N^2 << p = m$ , $N = 2^{16} = $
do $m=2^{50}$ at least!!

2. Clustering: if I take $k$ succesive random number and I plotted in
the N dimensional space I have to have a uniform distribution, not
clusters. $k-> m^{1/k}$ $2^{32/5} = 80$

3. Efficiency

4. repeatability, you would like to repeat the same sequence of random
numbers. This is important when you are debugging your
algorithm/code. 

5. Portability, Will the RNG always give you the same sequence no
matter what computer or OS you have.

Some times $a$ if it s very big would give an overflow, it is art how
to pick $a, c, m$ $m$ largest possible prime that you can register. $a$
relative prime numbers!

\section{Beyond LGM}

1. Non linear methods
$X_{n} = (a X^{2}_{n-1} + bX_{n-1}) + C$
2. Multipler recursive methods
$X_n = (a_1 X_{n-1} + .... + a_k X_{n-k})$


\end{document}
