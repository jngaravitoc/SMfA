\documentclass[11pt]{article}
\usepackage[hmargin=2.0cm,vmargin=1cm]{geometry}
\usepackage[utf8]{inputenc}
\usepackage{graphicx}
\usepackage{float}
\usepackage{cite}
\usepackage{natbib}
\usepackage{amsmath}

\title{\begin{LARGE}
{Correlations on the derivation of Cosmological parameters from Supernovae Ia
with the host galaxy mass.}
\end{LARGE}}
\begin{document}
\maketitle

\author{ \begin{center}
Ekta Patel, Carolyn Raithel \& J. Nicolas Garavito-Camargo
\end{center}}

\section{Introduction}

%Motivation
Since the late $20's$ the expansion Universe was observed by Edwin
Hubble, this observations proobe to Lemaitres theroy that states that
the Universe is expanding, this was the foundations of the Big Bang
Theory. Hubble measures relate the velocity at which galaxies are
reciding due to the expansion of the Universe with their distances,
this is know as the Hubble diagram. The slope of this relation known
as the Hubble constant is a direct measure of the expansion rate of
the Universe, this rate depends on the compositions of mass-energy in
the Universe Eq.\ref{eq:hubble}. Where $H_0$ is the value of the
Hubble constant at the present time, $z$ the redshift, $\Omega_m$
 is the amount of matter in the Universe which is mostly dark matter,
 in principle $\Omega_m$ decelerates the expansion of the Universe
 but this could be modified
by the energy content of the Universe $\Omega_{\Lambda}$, this form of energy
is believed to be the energy of the vacuum (Einstein
Cosmological constant) which acts as a negative pressure that do work
making the expansion of the Universe being accelerate. Measures of
objects at higher redshifts will allow to constrain more the evolution
of the cosmological parameters and then to distinguish between feasible
cosmological models.

\begin{equation}\label{eq:hubble}
H^2(z) = H_0^2 [\Omega_m(1+z)^3 + \Omega_k (1+z)^2 + \Omega_{\Lambda}]
\end{equation}

Type Ia supernovae are of particular interest in this endeavour since
they are the most brightest of all the supernovas types reaching an absolute magnitude
of -19.3, therefore at higher redshifts must of the observed
supernovas are Type Ia. Furthermore, their peak luminosity
present less dispersion than other types of supernovas.
In addition to that correlations of their photometric properties such
as the width of their light curves and the color with the peak
absolute magnitude reduces the dipsersion in the distance modulus
($\mu$) to $\sim 0.14$ mag. All these properties make Type Ia SN
an excellent cosmological distance indicators. With high-z Type Ia SN 
observations two research teams The high-z Supernova search
team lead by Adam Riess and Brian P. Schmidt and the Supernova
Cosmology project lead by Saul Perlmutter led to the discover that the
expansion of the Universe is accelerating Riess98, Perlmutter99.

The derivation of the cosmological parameters rely on fitting the
light curve of each SN Ia from the apparent magnitude measurements,
once the light curve is reconstructed the absolute magnitude and
consequently the distance modulus of each SNe can be derived. However,
recent works of Kelly2010 have revealed that there might be a correlation in
the absolute magnitude derived from the procedure explain above with
the host galaxy mass. In particular the mass of the host galaxy is
related with the dust extinction, stellar population age and
progenitor metallicity. Further investigation in this correlations
will lead to decrease the scatter in the distance modulus and
therefore a decrease in the error of the cosmological parameters
derived.

Quantifying this correlations is the main purpose of this project,
we explore the correlations between host of a SNe Ia 
galaxy mass with the derivation if the cosmological parameters.
To this aim we derive the cosmological parameters from SNe
measurements in different host galaxy masses.
In \S \ref{sec:methods} we review the methodology to
derive the cosmological parameters from the SNe distance modulus
measurements computed by Campbell2013. In \S
\ref{sec:assumptions} we explain the assumptions made in our derivations
and in the distance modulus computation made by Campbell. 
In \S \ref{sec:data} we explain the main characteristics of the data used in this
project which was obtained from Cambell2013.
 The main results are presented in \S \ref{sec:results} and the
discussions and conclusions are presented in \S \ref{sec:conclusions}.

% Why mass relation

% How this paper is organized

\section{Methods}{\label{sec:methods}}
%Methods

% Brief explantion of Supernovae Ia
%type Ia supernovae occurs when a white dwarf reaches the temperature
%to start a carbon fusion process  after reaching a mass limit due to
%the accretion of mass from a companion star.

From observations of the apparent magnitude at different times the light
curve of the SNe Ia can be reconstructed and therefore the absolute
magnitude at the peak can be derived see \S \ref{sec:assumptions} for
details. Therefore the distance modulus $\mu_0$ to each SN can be computed.
On the other hand from the Friedmann-Lemaitre-Robertson-Walker
cosmology the distance modulus $\mu_p$ can be derived as a function of the
luminosity distance see Eq.\ref{eq:dmodulus}.

%\begin{equation}
%_L = \left( \dfrac{L}{4\pi F}\right)^{1/2}
%\end{equation}

\begin{equation}\label{eq:dmodulus}
\mu_p =  5Log[D_L(z:\Omega_m, \Omega_{\Lambda})] + 25
\end{equation}

Where the luminosity distance is defined as:

\begin{equation}
D_L = cH_{0}^{-1} (1+z)|\Omega_k|^{-1/2} sin n\left[ |\Omega_k|^{-1/2}
\int_0^{\infty} dz[(1+z)^2(1+\Omega_Mz) - z(2+z)\Omega_{\Lambda}]^{-1/2}\right]
\end{equation}

Where $\Omega_k = 1 - \Omega_M - \Omega_{\Lambda}$ and $sin n$ is
$sinh$ when $\Omega_K \geq 0$ and $sin$ when $\Omega_K \leq 0$. The
data described in \S \ref{sec:data} already computes $\mu_0$ from XXX
method. With this data the corresponding likelihood is computed:

\begin{equation}
\chi^2(H_0, \Omega_M, \Omega_{\Lambda}) = \sum_i
\dfrac{[\mu_p - \mu_{0,i}]^2}{\sigma_{\mu_{0,i}} + \sigma_v^2}
\end{equation}

Where $\sigma_v$ is the associated uncertainty of the peculiar velocity
of the galaxy and $\sigma_{mu_{0,i}}$ the uncertainty of the observed
distance modulus.

We compute the $\chi^2$ as a function of the three free parameters
$H_0, \Omega_m, \Omega_{\Lambda}$ and then marginalize over $H_0$
to find contours in the $\Omega_m - \Omega_{\Lambda}$ plane.

\section{Assumptions}\label{sec:assumptions}

In this section we explain the methods us

\subsection{Light Curve Model}

\subsection{K-correction}

\subsection{Host Galaxy Mass}

\section{Data}\label{sec:data}

The data used in this project can be downloaded online in
\verb+http://www.mnras.oxfordjournals.org/lookup/suppl/doi:10.1093/mnras/stw115/-
/DC1+, this data was observed with the SDSS 2.5m telescope
\citep{Gunn98}. The photometry was taken by the SDSS-II-SN Survey
\citep{Frieman08, Sako08, Sako14,}, the survey scan the sky in
$300$ deg2 during three years 2005-2007, the survey took multicolor images ($u,g,r,i,z$) during
three months per year. Approximately 500 SNe Ia were found and have
spectroscopic confirmation. On the other hand, XXX candidates where
observed but doesn't have spectroscopic confirmation. However,
\citep{Campbell13} show that with an efficient photometric
calibration these objects might be used to constrain the cosmological
parameters. In order to perform such a calibration the redshift of
the SNe-Ia host galaxy has to be known, data from the BOSS survey
where used to this aim this allowing to a sample of 752 SNe Ia candidates
between redshift $[0.05 - 0.55]$.

The procedure to identify these candidates is based on a Bayesian light
curve classifier method (PSNID) \citep{Sako11}.
Bellow we briefly resumed the main aspects of this method:

The likelihood $\chi$ is computed for the observed photometry and
templates of SN Ia and core-collapse SN and identify the more likely
SN type. The algorithm also compute the probability that a SN
candidate could be either Ia, Ib/c or type II. The Bayesian evidence
is computed by marginilizing PDF over the parameter space:

\begin{equation}
E_{Ia} = \int P_{pr}(z, A_v, T_{max},\Delta m_{12,B},
\mu)e^{-\chi^2/2} dzdA_vdT_{max}d\Delta m_{15,B}d\mu
\end{equation}

Where $z$ is the redshift, $A_v$ the host galaxy extinction,
$T_{max}$ the time of maximum light, $\Delta m_{15,B}$ the amount of
magnitude decline in the B-band during the first 15 days after the
maximum light, and $\mu$ is the distance modulus. The method use flat
prior in $A_v, T_{max}, \Delta_m{12,B} \& \mu$, while in redshift it
assumes Gaussian priors. The evidece of non SNe Ia is given by:

\begin{equation}
  E_{Ibc, II} \sum \int P(z)e^{\chi^2/2}dz dA_{v}dT_{max}d\mu
\end{equation}

Then the probability that the light curve it's of a one
type of SN is given by:


\begin{equation}
 P_{type} = \dfrac{E_{type}}{E_{Ia} + E_{Ibc} + E_{II}}
\end{equation}

After selecting the SNe Ia the distance modulus is derived by fitting
the light curve as a function of the stretch ($x_1$) the colour
$c$ and
the apparent magnitude $m_B$. With these parameters the distance modulus can
be derived using:


\begin{equation}
\mu = m_B - M + \alpha x_1 + \beta c - \mu_{corr}
\end{equation}

Where $\alpha, \beta$ and $M$ (absolute magnitude in the B band)
are constants, that are going to be marginalized with flat priors.
The $\mu_{corr}$ term corresponds to the correction due to the
Malmquist bias correction defined as:

\begin{equation}
\mu_{corr} = ae^{(bz)} + c
\end{equation}

With $a=-0.004\pm 0.001, b=7.26\pm0.31$ and $c=0.004\pm0.006$.

Finally the host galaxy properties are also taken from the BOSS survey, From
the sample of 752 SNe Ia candidates AGN host galaxies have to be
removed from the sample because the line emission might dominate over
the line flux from star formation which is one of the properties that
\citep{\Campbell16} studied. The stellar mass of the remaining 581
host galaxies was derived by studying the broad band photometry and
comparing it with the best SED fit template. The chosen template used
a \citep{Maraston13} population synthesis model and a Kroupa IMF.


\section{Results}\label{sec:results}

\section{Conclusions}\label{sec:conclusions}

%\references

\end{document}

