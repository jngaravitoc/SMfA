\documentclass[11pt]{article}
\usepackage[hmargin=2.0cm,vmargin=1cm]{geometry}
\usepackage[utf8]{inputenc}
\usepackage{graphicx}
\usepackage{float}
\usepackage{cite}
\usepackage{natbib}
\usepackage{amsmath}

\title{\begin{LARGE}
{Dependence on the Cosmological Parameters derived from Supernovae Ia
with the host galaxy mass.}
\end{LARGE}}
\begin{document}
\maketitle

\author{ \begin{center}
Ekta Patel, Carolyn Raithel \& J. Nicolas Garavito-Camargo
\end{center}}

\section{Introduction}

%Motivation
Measures of the rate at which the Universe is expanding has been one of
the most important achievements in observational Cosmology in the last
century. In particular observations of Type Ia supernovae have revealed
that the Universe expansion is accelerating.

%n 1929 Edwin Hubble observe that galaxies that are at larger
%istance have are residing at larger velocities due to the
%expansion of the Universe. The slope of the distance velocity relationship is
%known as the Hubble constant and it's a measure of the rate at which the
%Universe is expanding. Thanks to the new technology more precise measurements have
%been done at larger redshifts is order to see if the rate of the
%expansion of the Universe (The Hubble constant) is constant across the
%cosmic History. Using observations of Supernovae Ia two teams XX and
%XX have measured that the 

\begin{equation}
H^2(z) = H_0^2 (\Omega_m(1+z)^3 + \Omega_k (1+z)^2 + \Omega_{\Lambda})
\end{equation}

In this project we explore the correlations between host SNe Ia mass
with the cosmological parameters. In \S \ref{mehods} we review the methodology to
derive the cosmological parameters from SNe measurements. In \S
\ref{assumptions} we explain the assumptions made in th SNe distance
and magnitude , In \S \ref{data} we explain the data sample in this
project. The main results are presented in \S \ref{results} and a
discussions and conclusions are presented in \S \ref{conclusions}.

% Why mass relation

% How this paper is organized

\section{Methods}{\label{methods}}
%Methods


% Brief explantion of Supernovae Ia
Type Ia supernovae occurs when a white dwarf reaches the temperature
to start a carbon fusion process  after reaching a mass limit due to
the accretion of mass from a companion star.

From observations of the apparent magnitude the light curve of the SNe
Ia can be reconstructed and the absolute magnitude can be derived.
The distance modulus to each SN is computed.

On the other hand the 
\begin{equation}
D_L = \left( \dfrac{L}{4\pi F}\right)^{1/2}
\end{equation}

\begin{equation}
m = M + 5Log[D_L(z:\Omega_m, \Omega_{\Lambda})] + K + 25
\end{equation}


\begin{equation}
D_L = cH_{0}^{-1} (1+z)|\Omega_k|^{-1/2} sin n\left[ |\Omega_k|^{-1/2}
\int_0^{\infty} dz[(1+z)^2(1+\Omega_Mz) - z(2+z)\Omega_{\Lambda}]^{-1/2}\right]
\end{equation}

Where $\Omega_k = 1 - \Omega_M - \Omega_{\Lambda}$

\begin{equation}
\mu_p = 5 Log D_L + 25
\end{equation}


\begin{equation}
\xi^2(H_0, \Omega_M, \Omega_{\Lambda}) = \sum_i
\dfrac{[\mu_p - \mu_{0,i}]^2}{\sigma_{\mu_{0,i}} + \sigma_v^2}
\end{equation}


% why mass relation



\section{Assumptions}\label{assumptions}

\subsection{Light Curve Model}

\subsection{K-correction}

\subsection{Host Galaxy Mass}

\section{Data}

\section{Results}

\section{Conclusions}

%\references

\end{document}

